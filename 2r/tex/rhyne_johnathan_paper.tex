\documentclass[review]{siamart220329}

% 1. Preamble and packages
\usepackage{lipsum}
\usepackage{amsfonts}
\usepackage{graphicx}
\usepackage{epstopdf}
\usepackage{algorithmic}
\ifpdf%
  \DeclareGraphicsExtensions{.eps,.pdf,.png,.jpg}
\else
  \DeclareGraphicsExtensions{.eps}
\fi
\usepackage{amsopn}
\DeclareMathOperator{\diag}{diag}
\usepackage{booktabs}

% 2. Paper title
\newcommand{\TheTitle}{%
  Improved DORGQR and Recursive DORG2R against optimized BLAS routines
}

% 2.5. Short title for running heads (if needed)
\newcommand{\TheShortTitle}{%
  \TheTitle
}

% 3. Student Name
\newcommand{\TheName}{%
  Johnathan Rhyne
}

% 4. Student Address
\newcommand{\TheAddress}{%
  University of Colorado Denver,
  (\email{johnathan.rhyne@ucdenver.edu}).
}

% 5. Acknowledge funding or other resources
\newcommand{\TheFunding}{%
  This work was funded by IJK\@.
}

% 6. Collaborators, such as advisor or research collaborators
\newcommand{\TheCollaborators}{%
  Julien Langou Advisor
}

% ---------------------------------------------
% ---------------------------------------------
\author{\TheName\thanks{\TheAddress}}
\title{{\TheTitle}\thanks{\TheFunding}}
\headers{\TheShortTitle}{\TheName}
\ifpdf%
\hypersetup{%
  pdftitle={\TheTitle},
  pdfauthor={\TheName}
}
\fi

% My commands
\newcommand{\dorgqr}{\texttt{dorgqr}\ }
\newcommand{\dorg}{\texttt{dorg2r}\ }
\newcommand{\dorgkr}{\texttt{dorgkr}\ }
\newcommand{\dlarft}{\texttt{dlarft}\ }

\begin{document}

\maketitle

\begin{center}
In collaboration with:
  {\TheCollaborators}
\end{center}
\vspace{1cm}
% ---------------------------------------------
% ---------------------------------------------

\begin{abstract}
    The LAPACK routine dorg2r is used to attain the full Q matrix in the QR decomposition, and is needed even when the blocked 
    version is used. Using a recursive scheme, we improve the performance of the dorg2r in the case of having number of 
    reflectors equal to the number of columns of Q. In addition, our scheme produces a way to compute The first $k$ columns of 
    Q only, which can be useful when coupled with a more efficient scheme to produce the remaining columns needed. We also 
    propose a new version of dorgqr where we also see performance increases for modestly sized inputs against both reference 
    LAPACK and tested optimized BLAS routines.
\end{abstract}

\begin{keywords}
  % 7. Keywords that describe the paper
    lapack
    blas
    recursive
    qr
    nullspace
\end{keywords}

\section{Introduction}\label{sec:intro}
The lapack routines \dorgqr and \dorg are two methods of computing the $Q$ matrix associated with the $QR$ 
decomposition of a matrix, denoted $A$, from the householder reflectors. The existing functionality in LAPACK does not 
properly take advantage of the fact that our first step in computing the columns between the desired number of columns of $Q$, 
denoted $n$, and the number of reflectors used in the decomposition of $A$, denoted $k$, involves a matrix multiplication
where one of the elements is the identity matrix. We exploit this fact.

\section{Main Contribution}\label{sec:main}
Our main contribution is an implementation of an improved \dorgqr. We do this by taking advantage of the first step in the \dorgqr
algorithm being a matrix multiplication by an identity matrix and blocking for the entire matrix instead of calling \dorg to 
do these first columns.

\section{Numerical Results}\label{sec:num}
Our numerical experiments test on the following hardware:
\begin{verbatim}
Insert processor information for laptop and the HPC. 
\end{verbatim}
We test 3 different versions of \dorgqr. 
\begin{enumerate}
    \item[] Version 1: Updated \dorgqr
    \item[] Version 2: Adding our new recursive \dorgkr as a helper routine
    \item[] Version 3: Adding our new recusrive \dlarft as a helper routine
\end{enumerate}
For Version 2, we also test \dorgkr against \dorg for performance that mimic the instances it is called from within dorgqr, so
that we can demonstrate the gain directly. This means that we do not time the call to \dlarft as this call is done regardless
inside \dorgqr. 

For Version 3, we perform the same timing as for Version 1, but also include timing just the calls to dlarft.

Our testing paradigm is as follows: 

We test only accuracy with the fortran tester routines. We have these Fortran testers to be able to see how accurate we are
under these conditions. However, we have a tester that lives only in \verb|C| that calls the fortran subroutines to measure
both accuracy and timing. We treat these as the main source for this write-up as it is the closest related to how most users
will interact with the LAPACK library. 

\section{Conclusions}\label{sec:conc}

\bibliographystyle{siamplain}
\bibliography{references}

\end{document}
