\documentclass{article}

\usepackage{mystyle}
\usepackage{algorithm}

\begin{document}
\tableofcontents
\section{What does Reference DORGQR do?}
The algorithm for the current Reference DORGQR is as follows. We will assume
for simplicity's sake that $n$ is a perfect multiple of $k$ and that we start
blocking exactly at $k$. This is not exactly how DORQR is currently written, 
but it will allow us to talk about the algorithm and ignore some technical 
difficulties in the implimentations
\begin{algorithm}
    \caption{Reference DORGQR}
    \begin{algorithmic}
        \REQUIRE $A\in\R^{m\times n}$ output of DEQRF. IE the $i^\text{th}$ column of $A$ is the vector defining the $i^\text{th}$ elementary reflector ($H_i$ for the starting matrix for $i=1,\dots,k$, and $m\geq n$
        \ENSURE $A = Q\in\R^{m\times n}$ such that $Q = H_1H_2\cdots H_k$
        \IF{blocking}
        \STATE determine blocking parameter $nb$
        \ELSE 
        \STATE $Q \gets$ DORG2R($A$)
        \RETURN $Q$
        \ENDIF
        \STATE Break $A$ down according to Figure \ref{fig:C1C2} where $C_1\in\R^{k\times n-k}$ and $C_2\in\R^{m-k\times n-k}$ are the last $k$ columns of $A$.
        \STATE $C_1 \gets \textbf{0}$
        \STATE $C_2 \gets$ DORG2R($C_2$)
        \FOR{$i=k-nb,k-2nb,\cdots 1$}
        \STATE Construct $T$ such that $H = I -VTV^\top$ where $H=H_iH_{i+1}\cdots H_{i+nb-1}$ 
        \STATE Apply
        \ENDFOR
    \end{algorithmic}
\end{algorithm}
\begin{figure}
    $$
    \begin{bmatrix}
        A_{11} & C_1 \\
        A_{21} & C_2
    \end{bmatrix}
    $$
    \caption{Decomposition of $A$ used in DORGQR}\label{fig:C1C2}
\end{figure}
\section{What did we aim to do?/Deliverables}
We aim to increase the performance of the reference DORGQR in both time and memory used
\section{Why do we care?}
\section{Hardware Used}
We ran the following tests using a Lenovo Thinkpad E430 running Arch Linux with the following system specifications
\begin{itemize}
    \item Kernel: 6.5.8-arch1-1
    \item CPU: Intel i3-3120M (4 cores, 8 threads) @ 2.500GHz
\end{itemize}
\section{Version 1}\label{sec:v1}
The file that contains just the changes mentioned here is: \verb|my_dorgqr_v1.f|
\subsection{Changes}
For the first version, we aimed to take advantage of the fact that in our first step, we have an identity matrix 
in the slot of $C_2$, and $0$ inside the slot of $C_1$ on every iteration.
\subsection{Numerical Performance}
We compare this version against two baselines. 
\begin{itemize}
    \item Reference DORGQR
    \item MKL DORGQR
\end{itemize}
\section{Version 2}
The file that contains the changes mentioned here and the ones described in Section \ref{sec:v1} is: \verb|my_dorgqr_v2.f|
\subsection{Changes}
\subsection{Numerical Performance}
\section{Summary}
\end{document}
